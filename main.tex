% This must be in the first 5 lines to tell arXiv to use pdfLaTeX, which is strongly recommended.
\pdfoutput=1
% In particular, the hyperref package requires pdfLaTeX in order to break URLs across lines.

\documentclass[11pt]{article}

\usepackage[a4paper, total={6in, 8in}]{geometry}

\usepackage{mystyle}

\usepackage{natbib}
\bibliographystyle{abbrvnat}
\usepackage{hyperref}

% Standard package includes
\usepackage{times}
\usepackage{latexsym}

% For proper rendering and hyphenation of words containing Latin characters (including in bib files)
\usepackage[T1]{fontenc}
% For Vietnamese characters
% \usepackage[T5]{fontenc}
% See https://www.latex-project.org/help/documentation/encguide.pdf for other character sets

% This assumes your files are encoded as UTF8
\usepackage[utf8]{inputenc}


\title{Applications of the Implicit Function Theorem}

\author{Justin Chiu \\
  Cornell Tech \\
  \texttt{jtc257@cornell.edu}}

\begin{document}
\maketitle
\begin{abstract}
Gradient-based learning forms the foundation of modern machine learning,
and automatic differentiation allows ML practitioners to easily compute gradients.
While automatic differentiation only costs a constant multiple of the time and space
required to evaluate a function, it has its limitations.
In particular, when evaluating a function itself is expensive,
the direct application of automatic differentiation is infeasible.
In this report, we review the implicit function theorem (IFT)
and its use in reducing the cost of computing gradients in scenarios where
function evaluation is expensive,
focusing on the application of implicit differentiation to variational inference.
\end{abstract}

\section{Introduction}
Gradient-based learning underpins many of the recent successes in machine learning,
particularly advances involving neural networks.
The key to the success of gradient-based methods is automatic differentiation (AD),
which has greatly increased the development speed of machine learning research by
allowing practitioners to circumvent the error-prone and time-consuming process
of computing gradients manually.
AD operates by reducing functions into compositions of atomic operations,
for which we have a library of derivatives for,
and composing those derivatives via the chain rule.
(introduce the representation of functions as evaluation procedures / computational graphs,
following \citet{griewank2008autodiff})
While efficient relative to the evaluation of the function in question,
taking only a multiplicative constant longer than the evaluation itself,
this may be prohibitively expensive if the original function evaluation itself is costly.
An example of this is if the function takes the form of an unrolled loop,
a common artifact of iterative methods.
As naive AD requires storing all of the intermediate values at each point,
storing the output of all computations at every iteration of a loop can quickly
become infeasible due to memory limitations.

There are a variety of methods for overcoming the space limitations of AD,
of which we only mention three: checkpointing, reversible computation, and implicit differentiation.
A first method, checkpointing, improves space complexity at the cost of time.
Rather than storing all intermediate computation,
checkpointing instead recomputes values when needed.
This can result in a large slowdown,
and also requires careful choosing of which computationals subgraphs to checkpoint.
A second method is an improvement upon checkpointing, called reversible computation \citep{maclaurin2015reversible,gomez2017reversible},
which improves space complexity at the cost of expressivity, but not speed.
Reversible computation ensures that the gradient with respect to input depends only on the output,
allowing the input to be discarded during function evaluation.
This is typically accomplished by ensuring that the input is easily reconstructed from the output,
restricting the expressivity of layers.
A third method is implicit differentiation,
which improves space complexity at the cost of stronger assumptions.
Implicit differentiation relies on the implicit function theorem (IFT),
which gives conditions under which derivatives can be computed independent of
intermediate computation.
Implicit differentiation requires a series of equations specified by a relation
In this report, we will cover the use of the implicit function theorem
in OptNet \citep{optnet}, which allows us to use the output of an
optimization problem inside a neural network.
%and apply it to reducing the computational cost of taking derivatives through variational inference.

\paragraph{Bilevel Optimization}
One application of implicit differentiation is bilevel optimization.
Bilevel optimization problems are, as implied by the name,
optimization problems with another nested inner optimization problem embedded within.
Methods for solving bilevel optimization typically proceed iteratively.
For every iteration when solving the outer optimization problem,
we must additionally solve an inner optimization problem.
Some applications that can be formalted as bilevel optimization problems are
hyperparameter optimization, metalearning, and variational inference.

Hyperparameter optimization formulates hyperparameter tuning, such as the shrinkage penalty in Lasso,
as a bilevel optimization problem by computing gradients wrt the penalty through the entire learning procedure
of the linear model \citep{lorraine2019implasso}.
(Other works on hyperparam opt \citep{maclaurin2015reversible,bertrand2020implicit})
Similarly, metalearning learns the parameters of a model such that is the model is able to quickly
be adapted to a new task via gradient descent \citep{finn2017maml,rajeswaran2019imaml}.
This is accomplished by differentiating through the learning procedure of each new task.
Finally, a variant of variational inference follows a very similar format:
semi-amortized variational inference (SAVI) aims to learn a model that is able to initialize
variational parameters \citep{kim2018savi}.
This is also accomplished by differentiating through the iterative optimization procedure
applied to the variational parameters during inference.
(Other VI papers \citep{vi,johnson2017pgm})

There is also work on expressing individual layers of a neural network declaratively
as the solution of an optimization problem \citep{optnet,agrawal2019diffcvx,gould2019declarative}.
This also falls under the umbrella of bilevel optimization, as we have both the outer training loop
and the inner optimization loop for each OptNet layer.

\section{The Implicit Function Theorem}
The Implicit Function Theorem (IFT) has a long history, as well as many applications
in a wide variety of fields such as economics and differential geometry.
For an overview of the history of the IFT, see the book by \citet{iftbook}.

The IFT gives sufficient conditions under which the solution $x$
to a system of equations, $F(\theta, x) = 0$ with $F: \R^n\times\R^m\to\R^m$,
can locally be written as a function of just the parameters $\theta$,
i.e. there exists a solution mapping $x^*$
such that $f(\theta, x^*(\theta)) = 0$ in the neighbourhood of the particular point
$\theta\in\dom(F)$.
These conditions are as follows:
\begin{enumerate}
\item We have a solution point $(\theta, x)$ that satisfies the system of equations
    $F(\theta, x) = 0$.
\item $F$ has at least continuous first derivatives: $F \in \mcC^k$.
\item The Jacobian of $F$ wrt $x$ evaluated at the solution point $(\theta,x)$ is nonsingular:
    $\det \frac{\partial F}{\partial x} \neq 0$.
\end{enumerate}
Given these conditions, we are able to assert the existence of the solution mapping $x^*(\theta)$,
and determine its derivative
$\frac{\partial x^*(\theta)}{\partial \theta} = -[\frac{\partial F(\theta,x)}{\partial x}]^{-1}
    \frac{\partial F(\theta,x)}{\partial \theta}$.
Rather than directly applying this formula,
we can use implicit differentiation to compute $\frac{dx^*(\theta)}{d\theta}$
which is more convenient when performing computation by hand (as we will do).

While the IFT has a long history and many applications, we will focus on one particular application:
We will use the solution to an optimization problem the output of a layer within a neural network,
following OptNet \citep{optnet}.
We will then discuss the problem addressed by this method in OptNet.
Afterwards, we will cover an application of the IFT to speed up variational inference.

\section{Embedding Optimization inside a Neural Network}
As an introductory example,
we will replace the softmax layer of a neural network with an equivalent function
defined as the output of an optimization problem, then derive derivatives using the IFT.
We will start by reviewing softmax and its expression as an optimization problem.
After checking the conditions of the IFT hold, we can then compute gradients.
Since the Jacobian of softmax is known, we can directly verify that the IFT gives
the correct answer.


\subsection{Softmax}
Softmax is often used to parameterize categorical distributions within neural networks,
such as in attention layers.
It has its origins in statistical mechanics and decision theory, and functions
as a differentiable surrogate for argmax.

Softmax assumes that we have $n$ items with independent utilities, $\theta \in \R^n$,
which indicate preferences.
Softmax then gives the following distribution over these items:
$p(x) = \frac{\exp(\theta_x)}{\sum_y \exp(\theta_y)}$.
Interestingly, softmax arises as the solution of an optimization problem
\citep{gao2018properties}.

The output of softmax is the solution of the following optimization problem:
\begin{equation}
\label{eqn:softmax-opt}
\begin{aligned}
\textrm{maximize } \quad & x^\top\theta + H(x)\\
\textrm{subject to } \quad & x^\top \mathbf{1} = 1\\
& x \succeq 0,
\end{aligned}
\end{equation}
where $H(x) = -\sum_i x_i \log x_i$ is the entropy.
This corresponds to an entropy-regularized argmax optimization problem.
We will refer to this as the softmax problem.

Our goal is to compute the Jacobian of softmax
using the IFT and the optimization problem above.
While this is not of practical use (there is a closed-form equation
for both softmax and its Jacobian),
we use it as an introduction to the mechanism
behind OptNet and differentiable optimization layers
\citep{optnet,agrawal2019diffcvx}.
Applying the IFT consists of three steps:
\begin{enumerate}
\item Write down the system of equations.
\item Check that the conditions of the IFT hold.
\item Compute the derivative of the implicit solution mapping wrt the parameters.
\end{enumerate}

\subsection{KKT Conditions}
Given an optimization problem, the KKT conditions determine a system of equations
that the solution must satisfy \citep{kkt-thesis,kkt}.
We will use the KKT conditions of the softmax problem in
Eqn.~\ref{eqn:softmax-opt} to determine $F(\theta,x)$ in the IFT.

First, we introduce dual variables $u\in\R,v\in\R^n$ and write out the Lagrangian:
$$\mcL(\theta, x, u, v) = x^\top\theta + H(x) + u(x^\top \mathbf{1} - 1) + v^\top x.$$
While we should technically include $u,v$ as solution variables,
we are mainly concerned with the primal variable $x$ and parameters $\theta$,
so we will be a bit loose with the word `solution'.
We then have the following necessary conditions for a solution $x$,
i.e. the KKT conditions:
\begin{equation}
\begin{aligned}
\nabla_x \mcL(\theta, x,u,v) = 0 && \textrm{(stationarity)}\\
u(x^{\top} \mathbf{1} - 1) = 0 && \textrm{(primal feasibility)}\\
\diag(v)x = 0 && \textrm{(complementary slackness)}\\
v \succeq 0 && \textrm{(dual feasibility)}
\end{aligned}
\end{equation}
As we are interested in the primal variable or solution $x$,
we focus on the first three conditions.

In full, the system of equations $F(\theta, x) = 0$ is
\begin{equation}
\label{eqn:system}
\begin{aligned}
\theta + -\log(x) - 1 + u\mathbf{1} + v &= 0\\
u(x^{\top} \mathbf{1} - 1) &= 0\\
\diag(v)x &= 0.
\end{aligned}
\end{equation}

Now we can check the conditions of the IFT.
Any solution $x$ will satisfy $F(\theta, x) = 0$,
and $F \in \mcC^1$.
All that remains is to check that the Jacobian matrix of $F$ is non-singular.

Taking the differential of $F(\theta,x,u,v) = 0$ yields
\begin{equation}
\label{eqn:dsystem}
\begin{aligned}
d\theta - \frac{dx}{x} + du\mathbf{1} + dv &= 0\\
du(x^\top \mathbf{1} - 1) + u\mathbf{1}dx &= 0\\
\diag(v)dx + \diag(x)dv &= 0.
\end{aligned}
\end{equation}
Rearranging into matrix form and separating the solution variables $x,u,v$
from the parameters $\theta$, we have
\begin{equation}
\label{eqn:dsystem-matrix}
\begin{bmatrix}
\diag(x)^{-1} & -\mathbf{1} & -I_n \\
u\mathbf{1}^\top & x^\top\mathbf{1} - 1 & 0\\
\diag(v) & 0 & \diag(x)
\end{bmatrix}
\begin{bmatrix}
dx \\ du \\ dv
\end{bmatrix}
=   
\begin{bmatrix}
d\theta \\ 0 \\ 0
\end{bmatrix},
\end{equation}
giving us the Jacobian matrix of $F$ wrt the solution variables.
Since a solution must be feasible, we know that $x^\top\mathbf{1} = 1$ and $u > 0$.
With the additional information that the domain of $H(x)$
adds the implicit constraint that $\forall i, x_i > 0$,
we can deduce that the Jacobian of $F$ is full rank and therefore has nonzero determinant.
This shows that the conditions of the IFT hold.

It is important to note that the stationarity condition already used the Jacobian of
the objective.
Further differentiating the stationarity condition leads to the Hessian,
which is much larger than the Jacobian:
the Jacobian of a scalar function (i.e. the Lagrangian) is a vector in $\R^n$,
while the Hessian is a matrix in $\R^{n \times n}$,
where $n$ is the number of variables.
In the case of softmax we will show that we
can avoid explicitly constructing the Hessian and compute the
gradient analytically.
However, in a majority of other cases some form of Hessian (or Hessian-vector-product)
approximation will be needed \citep{rajeswaran2019imaml,lorraine2019hoift}.

\subsection{The Jacobian of Softmax}
Now that we have shown that the conditions of the IFT hold,
we can proceed to apply the second part of the IFT.
The second part of the IFT tells us that we can compute the Jacobian of the
solution mapping $\frac{dx^*(\theta)}{d\theta}$ via implicit differentiation.

This is accomplished by solving the system of equations above in Eqn.~\ref{eqn:system}
for the entries of the upper-left $n\times n$ block corresponding to $\frac{dx}{d\theta}$,
i.e. 
\begin{equation}
\label{eqn:dsystem-matrix}
\begin{bmatrix}
\diag(x)^{-1} & -\mathbf{1} & -I_n \\
u\mathbf{1}^\top & x^\top\mathbf{1} - 1 & 0\\
\diag(v) & 0 & \diag(x)
\end{bmatrix}^{-1}
= 
\begin{bmatrix}
\frac{dx}{d\theta} & \cdots \\
\vdots & \ddots
\end{bmatrix}.
\end{equation}
We use the block-wise inversion formula
\begin{equation*}
\begin{bmatrix}
A & B\\
C & D
\end{bmatrix}^{-1} = \begin{bmatrix}
    (A - BD^{-1}C)^{-1} & 0\\
    0 & (D - CA^{-1}B)^{-1}
\end{bmatrix}
\begin{bmatrix}
    I & -BD^{-1}\\
    -CA^{-1} & I
\end{bmatrix},
\end{equation*}
where
\begin{align*}
A = \begin{bmatrix} \diag(x)^{-1} & -\mathbf{1} \\ u\mathbf{1}^\top & 0 \end{bmatrix}&\qquad\qquad
B = \begin{bmatrix}-I_n \\ 0\end{bmatrix}\\
C = \begin{bmatrix}\diag(v) & 0\end{bmatrix} &\qquad\qquad
D = \diag(x).
\end{align*}
However, by complementary slackness, we have $v = 0$, reducing the above to
\begin{equation*}
\begin{bmatrix}
A & B\\
C & D
\end{bmatrix}^{-1} = \begin{bmatrix}
    A^{-1} & 0\\
    0 & D^{-1}
\end{bmatrix}
\begin{bmatrix}
    I & -BD^{-1}\\
    0 & I
\end{bmatrix}.
\end{equation*}
As we are only interested in $\frac{dx}{d\theta}$,
we only have to solve for the upper-left $n\times n$ block of $A^{-1}\in\R^{n+1\times n+1}$.
To do so, we will repeat the same block-wise inverse computation.
Let us denote $A = \begin{bmatrix}E & F \\ G & H\end{bmatrix}$.
First, we compute the Schur complement of $A$,
\begin{equation}
A/E = H - GE^{-1}F = 0 + u\mathbf{1}^\top\diag(x)\mathbf{1} = ux^\top\mathbf{1}.
\end{equation}
Since $x$ is feasible, we have $A/E = u$ due to the equality constraints
($x$ must sum to 1 as a probability mass function).
Then, we have
\begin{equation}
A^{-1} = \begin{bmatrix}
\diag(x)^{-1} & -\mathbf{1}\\
u\mathbf{1}^\top & 0
\end{bmatrix}^{-1}
=\begin{bmatrix}E & F \\ G & H\end{bmatrix}^{-1}
= \begin{bmatrix}
E^{-1} + E^{-1}F(A/E)^{-1}GE^{-1} & -E^{-1}F(A/E)^{-1}\\
-(A/E)^{-1}GE^{-1} & (A/E)^{-1}
\end{bmatrix}.
\end{equation}
Plugging in, we have
\begin{equation}
\begin{aligned}
A^{-1} 
&= \begin{bmatrix}
\diag(x) - \diag(x)\mathbf{1}u^{-1}u\mathbf{1}^\top\diag(x)
    & \diag(x)\mathbf{1}u^{-1}\\
-u^{-1}u\mathbf{1}^\top \diag(x) & u^{-1}
\end{bmatrix}\\
&= \begin{bmatrix}
\diag(x) - xx^\top
    & u^{-1}x\\
    -x^\top & u^{-1}
\end{bmatrix}.
\end{aligned}
\end{equation}
Pulling out the top-left $n\times n$ block yields
the Jacobian $\frac{\partial x}{\partial \theta} = \diag(x) - xx^\top$,
which agrees with directly differentiating softmax \citep{sparsemax}.

With this, we have shown that we can rewrite softmax as an optimization problem,
and differentiate the solution of that problem wrt the parameters in a solver-agnostic
manner.

(Example showing build-up of memory from reversible SGD vs IFT would be nice here,
ie code it up and plot memory consumption / speed. maybe not best example though)

While softmax has an explicit functional form that determines the relationship between
the parameters $\theta$ and solution $x$, the IFT applies when
the relationship between parameters and solutions is not explicit.

\section{OptNet}
OptNet generalizes the methodology applied above to the softmax problem by
including parameterized constraints and computing the gradients of the solution
wrt all parameters.
This allows us to learn not only the objective, but also the constraints.
Constraint learning has the potential to 

\subsection{Quadratic Programs}
OptNet applies the IFT to quadratic programs in particular.
The methodology remains the same: given a quadratic program (QP) and a solution,
use the KKT conditions to produce a system of equations then apply the IFT
/ implicit differentiation to compute the derivative of the solution wrt the
parameters of the objective and constraints.

Quadratic programs take the following form:
\begin{equation}
\label{eqn:qp}
\begin{aligned}
\textrm{maximize } \quad & \frac12 x^\top Q x + q^\top x\\
\textrm{subject to } \quad & Ax = b\\
& Gx \leq h,
\end{aligned}
\end{equation}
where we optimize over $x$ and the parameters are $\theta = \set{Q, q, A, b, G, h}$.

\section{Semi-Amortized Variational Inference (POSTPONED)}
Variational inference has found success in recent applications to generative models,
in particular by allowing practitioners to depart from conjugate models
and extend emission models with expressive neural network components.
The main insight that led to this development is that inference can be amortized through
the use of an inference network.
One approach to variational inference, stochastic variational inference (SVI),
introduces local, independent variational parameters for every instance of hidden variable.
While flexible, the storage of all variational parameters is expensive, and the optimization
of each parameter independently slow \citep{}.
Amortized variational inference (AVI) solves that by instead sharing variational parameters hierarchically
via an inference network, which in turn generates the local variational parameters \citep{}.
The resulting local parameters may or may not be subsequently optimized.

Failure to further optimize may result in an amortization gap \citep{}.
Prior work has shown that this gap can be ameliorated by performing a few steps of
optimization on the generated local parameters obtained from the inference network,
and even by propagating gradients through the optimization process.
Optimizing through the inner optimization problem results in semi-amortized variational inference
(SAVI) \citep{}.

As our main motivating example, we will examine whether we can apply the IFT to SAVI.
We will start by formalizing the problem of variational inference for a simple model.

We will start with a model defined by the following generative process,
used by \citet{dai2020vae} to analyze posterior collapse:
\begin{enumerate}
\item Choose a latent code from the prior distribution $z \sim p(z) = N(0, I)$.
\item Given the code, choose an observation from the emission distribution
    $x \mid z \sim p_\theta(x \mid z) = N(\mu_x(z, \theta), \gamma I)$,
\end{enumerate}
where $\mu_x(z, \theta) \equiv \MLP(z, \theta)$ and $\gamma > 0$ is a hyperparameter.
This yields the joint distribution $p(x,z) = p(x\mid z)p(z)$.

Since the latent code $z$ is unobserved, training this model would require optimizing the
evidence $p(x) = \int p(x,z)$.
However, due to the MLP parameterized $\mu_x$, the integral is intractable.
Variational inference performs approximate inference by introducing variational distribution
$q_\phi(z \mid x)$ and maximizing the following lower bound on $\log p(x)$:
\begin{equation}
    \log p(x) - \KL{q(z \mid x) || p(z \mid x)}
    = \Es{q_\phi(z \mid x)}{\log \frac{p_\theta(x,z)}{q_\phi(z\mid x)}} = \mcL(\theta, \phi).
\end{equation}

(Write out objective in full.)

While SVI introduces local parameters for each instance of $z$,
and AVI uses a single $q(z \mid x)$ for all instances,
we will follow the approach of SAVI.
We will perform inference as follows:
For each instance $x$, produce local variational parameter
$z^{(0)} = g(x; \phi)$.
Obtain $z^*$ by solving $\mcL(\theta, z^{(0)}) = 2$, with (local) optima $\ell^*$.
Take gradients through the whole procedure,
i.e. compute $\frac{\partial \ell^*}{\partial \phi}
= \frac{\partial\ell^*}{\partial z^*}\frac{\partial z^*}{\partial z^{(0)}}
\frac{\partial z^{(0)}}{\partial \lambda}$.
The main difficuty lies in computing $\frac{\partial z^*}{\partial z^{(0)}}$.
(Highlight challenge)

In order to avoid the memory costs of storing all intermediate computation performed
in a solver, we will instead apply the IFT.
In order to apply the IFT, we must satisfy the three conditions.
First, we must have a solution point to a system of equations, $F(x_0, z_0) = 0$.
In this setting, we will use the KKT conditions of the optimization problem to define $F$.

\section{Limitations}

\bibliography{bib}

\appendix

\section{Example Appendix}
\label{sec:appendix}

Neural ODEs use reversibility.

\end{document}
