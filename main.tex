% This must be in the first 5 lines to tell arXiv to use pdfLaTeX, which is strongly recommended.
\pdfoutput=1
% In particular, the hyperref package requires pdfLaTeX in order to break URLs across lines.

\documentclass[11pt]{article}

\usepackage[a4paper, total={6in, 8in}]{geometry}

\usepackage{mystyle}

\usepackage{natbib}
\bibliographystyle{abbrvnat}
\usepackage{hyperref}

% Standard package includes
\usepackage{times}
\usepackage{latexsym}

% For proper rendering and hyphenation of words containing Latin characters (including in bib files)
\usepackage[T1]{fontenc}
% For Vietnamese characters
% \usepackage[T5]{fontenc}
% See https://www.latex-project.org/help/documentation/encguide.pdf for other character sets

% This assumes your files are encoded as UTF8
\usepackage[utf8]{inputenc}


\title{Implicit Differentiation}

\author{Justin Chiu \\
  Cornell Tech \\
  \texttt{jtc257@cornell.edu}}

\begin{document}
\maketitle
\begin{abstract}
Gradient-based learning forms the foundation of modern machine learning,
and automatic differentiation allows ML practitioners to easily compute gradients.
While automatic differentiation only costs a constant multiple of the time and space
required to evaluate a function, it has its limitations.
In particular, when evaluating a function itself is expensive,
the direct application of automatic differentiation is infeasible.
In this report, we review the implicit function theorem (IFT)
and its use in reducing the cost of computing gradients in scenarios where
function evaluation is expensive,
focusing on the application of implicit differentiation to variational inference.
\end{abstract}

\section{Introduction}
Gradient-based learning underpins many of the recent successes in machine learning,
particularly advances involving neural networks.
The key to the success of gradient-based methods is automatic differentiation (AD),
which has greatly increased the development speed of machine learning research by
allowing practitioners to circumvent the error-prone and time-consuming process
of computing gradients manually.
AD operates by reducing functions into compositions of atomic operations,
for which we have a library of derivatives for,
and composing those derivatives via the chain rule.
(introduce the representation of functions as evaluation procedures / computational graphs,
following \citet{griewank2008autodiff})
While efficient relative to the evaluation of the function in question,
taking only a multiplicative constant longer than the evaluation itself,
this may be prohibitively expensive if the original function evaluation itself is costly.
An example of this is if the function takes the form of an unrolled loop,
a common artifact of iterative methods.
As naive AD requires storing all of the intermediate values at each point,
storing the output of all computations at every iteration of a loop can quickly
become infeasible due to memory limitations.

There are a variety of methods for overcoming the space limitations of AD,
of which we only mention three: checkpointing, reversible computation, and implicit differentiation.
A first method, checkpointing, improves space complexity at the cost of time.
Rather than storing all intermediate computation,
checkpointing instead recomputes values when needed.
This can result in a large slowdown,
and also requires careful choosing of which computationals subgraphs to checkpoint.
A second method is an improvement upon checkpointing, called reversible computation \citep{maclaurin2015reversible,gomez2017reversible},
which improves space complexity at the cost of expressivity, but not speed.
Reversible computation ensures that the gradient with respect to input depends only on the output,
allowing the input to be discarded during function evaluation.
This is typically accomplished by ensuring that the input is easily reconstructed from the output,
restricting the expressivity of layers.
A third method is implicit differentiation,
which improves space complexity at the cost of stronger assumptions.
Implicit differentiation relies on the implicit function theorem (IFT),
which gives conditions under which derivatives can be computed independent of
intermediate computation.
Implicit differentiation requires a series of equations specified by a relation
In this report, we will cover the use of the implicit function theorem
in OptNet \citep{}, which allows us to use the output of an
optimization problem inside a neural network.
%and apply it to reducing the computational cost of taking derivatives through variational inference.

\paragraph{Bilevel Optimization}
One application of implicit differentiation is bilevel optimization.
Bilevel optimization problems are, as implied by the name,
optimization problems with another nested inner optimization problem embedded within.
Methods for solving bilevel optimization typically proceed iteratively.
For every iteration when solving the outer optimization problem,
we must additionally solve an inner optimization problem.
Some applications that can be formalted as bilevel optimization problems are
hyperparameter optimization, metalearning, and variational inference.

Hyperparameter optimization formulates hyperparameter tuning, such as the shrinkage penalty in Lasso,
as a bilevel optimization problem by computing gradients wrt the penalty through the entire learning procedure
of the linear model \citep{lorraine2019implasso}.
(Other works on hyperparam opt \citep{maclaurin2015reversible,bertrand2020implicit})
Similarly, metalearning learns the parameters of a model such that is the model is able to quickly
be adapted to a new task via gradient descent \citep{finn2017maml,rajeswaran2019impmaml}.
This is accomplished by differentiating through the learning procedure of each new task.
Finally, a variant of variational inference follows a very similar format:
semi-amortized variational inference (SAVI) aims to learn a model that is able to initialize
variational parameters \citep{kim2018savi}.
This is also accomplished by differentiating through the iterative optimization procedure
applied to the variational parameters during inference.
(Other VI papers \citep{vi,johnson2017pgm})

There is also work on expressing individual layers of a neural network declaratively
as the solution of an optimization problem \citep{optnet,agrawal2019diffcvx,gould2019declarative}.
This also falls under the umbrella of bilevel optimization, as we have both the outer training loop
for the whole model and the inner optimization loop for each optimization layer.

\section{The Implicit Function Theorem}
The Implicit Function Theorem (IFT) has a long history, as well as many applications.
For an overview of the history of the IFT, see the book by \citet{iftbook}.

The IFT gives sufficient conditions under which a system of equations the solution $z^*$
to a system of equations, $F(x, z) = 0$ with $F: \R^n\times\R^m\to\R^m$,
can locally be written as a function of just $x$, i.e. there exists a solution function $z^*$
such that $f(x_0, z^*(x_0)) = 0$ in the neighbourhood of a given $x_0\in\dom F$.
These conditions are as follows:
\begin{enumerate}
\item We have a solution point $(x_0, z_0)$ that satisfies the system of equations
    $F(x_0, z_0) = 0$.
\item $F$ has at least continuous first derivatives: $F \in \mcC^k$.
\item The Jacobian of $F$ wrt $z$ evaluated at the solution point $(x_0,z_0)$ is nonsingular:
    $\det \frac{\partial F}{\partial z} \neq 0$.
\end{enumerate}
Given these conditions, we are able to assert the existence of the solution function $z^*(x)$,
and determine its derivative
$\frac{\partial z^*}{\partial x} = -[\frac{\partial F(x,z)}{\partial z}]^{-1}
    \frac{\partial F(x,z)}{\partial x} \Bigr\rvert_{(x,z) = (x_0,z_0)}$

While the IFT has a long history and many applications, we will focus on one particular application,
where we embed an optimization problem within a neural network,
following OptNet \citep{optnet}.
It is under this context that we will walk through the conditions and implications of the IFT.

After this, we will cover an application of the IFT to speed up variational inference.

\section{Embedding Optimization inside a Neural Network}
As a first example, we will replace the softmax layer of a neural network with an equivalent function
defined as the output of an optimization problem, and derive gradients using the IFT.
Applying the IFT in this manner consists of three steps:
\begin{enumerate}
\item Check the conditions of the IFT.
\item Compute gradients without explicitly computing the Jacobian
    (i.e. only compute the Jacobian-vector product).
\end{enumerate}

We will start by reviewing softmax and its expression as an optimization problem,
checking the conditions of the IFT hold, then compute gradients efficiently.

\subsection{Softmax}
Softmax is often used to parameterize categorical distributions within neural networks.
It is used to parameterize categorical outputs and attention.
It has its origins in statistical mechanics as well as decision theory, and functions
as a differentiable surrogate for argmax.

Softmax assumes that we have $n$ items with independent utilities, $\theta \in \R^n$,
which indicate preferences.
Softmax then gives the following distribution over these items:
$p(x) = \frac{\exp(\theta_x)}{\sum_y \exp(\theta_y)}$.
Interestingly, this arises as the solution of an optimization problem as well \citep{gao2018properties}.

We can formulate argmax over independent utilities as a linear program, namely
\begin{equation}
\begin{aligned}
\textrm{maximize } \quad & x^\top\theta\\
\textrm{subject to } \quad & x^\top \mathbf{1} = 1\\
& x \succeq 0.
\end{aligned}
\end{equation}
Without ties in the utilities $\theta$, the solution $x^*$ to this problem will be a
one-hot corresponding to the index of the highest item, i.e. the argmax.

Softmax results from solving a regularized version of the above optimization problem:
\begin{equation}
\label{eqn:softmax-opt}
\begin{aligned}
\textrm{maximize } \quad & x^\top\theta + H(x)\\
\textrm{subject to } \quad & x^\top \mathbf{1} = 1\\
& x \succeq 0,
\end{aligned}
\end{equation}
where $H(x) = \sum_i x_i \log x_i$ is the entropy.

Our goal will be to rewrite softmax using the IFT and the optimization problem in
Eqn.~\ref{eqn:softmax-opt}.
While this is not of practical use since there is a closed-form equation
for softmax, it is a good exercise in applying the methodology
used in OptNet and differentiable convex optimization layers \citep{optnet,agrawal2019diffcvx}.

\subsection{KKT Conditions}
Given an optimization problem, the KKT conditions determine a system of equations
that the solution must satisfy \citep{kkt-thesis,kkt}.
First, we introduce dual variables $u,v$ and write out the Lagrangian:
$$\mcL(\theta, x, u, v) = x^\top\theta + H(x) + u(x^\top \mathbf{1} - 1) + v^\top x.$$
We then have the following necessary conditions for a solution $x^*$:
\begin{equation}
\begin{aligned}
\nabla_x \mcL(\theta, x^*,u,v) = 0 && \textrm{(stationarity)}\\
\diag(v)x^* = 0 && \textrm{(complementary slackness)}\\
u(x^{*\top} \mathbf{1} - 1) = 0 && \textrm{(primal feasibility)}\\
v > 0 && \textrm{(dual feasibility)}
\end{aligned}
\end{equation}
As we are interested in the solution $x^*$, we focus on the first three conditions
which give us the system of equations $F(\theta, x)$ in the IFT.

In full, the system of equations $F(\theta, x) = 0$ is
\begin{equation}
\label{eqn:system}
\begin{aligned}
F_1(\theta,x,u,v) = \theta + \log(x) + 1 + u\mathbf{1} + v &= 0\\
F_2(\theta,x,u,v) = u(x^{\top} \mathbf{1} - 1) &= 0\\
F_3(\theta,x,u,v) = \diag(v)x &= 0.
\end{aligned}
\end{equation}

Now, we can check the conditions of the IFT.
Clearly, any solution $x^*$ must satisfy $F(\theta, x^*) = 0$,
and $F \in \mcC^1$.
We must then check that the Jacobian matrix is non-singular,
as well as compute the derivative
$\frac{\partial x}{\partial \theta} = -[\frac{\partial F(\theta,x)}{\partial x}]^{-1}
    \frac{\partial F(\theta,x)}{\partial z}$.

We use implicit differentiation, a method that follows from the IFT.
Taking the differential of $F(\theta,x,u,v) = 0$ yields
\begin{equation}
\label{eqn:dsystem}
\begin{aligned}
d\theta + \frac{dx}{x} + du\mathbf{1} + dv &= 0\\
du(x^\top \mathbf{1} - 1) + u\mathbf{1}dx &= 0\\
\diag(v)dx + \diag(x)dv &= 0.
\end{aligned}
\end{equation}
Rearranging into matrix form,
\begin{equation}
\label{eqn:dsystem-matrix}
\begin{bmatrix}
\diag(x)^{-1} & \mathbf{1} & \eye(n) \\
u\mathbf{1}^\top & x^\top\mathbf{1} - 1 & 0\\
\diag(v) & 0 & \diag(x)
\end{bmatrix}
\begin{bmatrix}
dx \\ du \\ dv
\end{bmatrix}
=   
\begin{bmatrix}
-d\theta \\ 0 \\ 0
\end{bmatrix}
\end{equation}

We then have that 

\section{Semi-Amortized Variational Inference (POSTPONED)}
Variational inference has found success in recent applications to generative models,
in particular by allowing practitioners to depart from conjugate models
and extend emission models with expressive neural network components.
The main insight that led to this development is that inference can be amortized through
the use of an inference network.
One approach to variational inference, stochastic variational inference (SVI),
introduces local, independent variational parameters for every instance of hidden variable.
While flexible, the storage of all variational parameters is expensive, and the optimization
of each parameter independently slow \citep{}.
Amortized variational inference (AVI) solves that by instead sharing variational parameters hierarchically
via an inference network, which in turn generates the local variational parameters \citep{}.
The resulting local parameters may or may not be subsequently optimized.

Failure to further optimize may result in an amortization gap \citep{}.
Prior work has shown that this gap can be ameliorated by performing a few steps of
optimization on the generated local parameters obtained from the inference network,
and even by propagating gradients through the optimization process.
Optimizing through the inner optimization problem results in semi-amortized variational inference
(SAVI) \citep{}.

As our main motivating example, we will examine whether we can apply the IFT to SAVI.
We will start by formalizing the problem of variational inference for a simple model.

We will start with a model defined by the following generative process,
used by \citet{dai2020vae} to analyze posterior collapse:
\begin{enumerate}
\item Choose a latent code from the prior distribution $z \sim p(z) = N(0, I)$.
\item Given the code, choose an observation from the emission distribution
    $x \mid z \sim p_\theta(x \mid z) = N(\mu_x(z, \theta), \gamma I)$,
\end{enumerate}
where $\mu_x(z, \theta) \equiv \MLP(z, \theta)$ and $\gamma > 0$ is a hyperparameter.
This yields the joint distribution $p(x,z) = p(x\mid z)p(z)$.

Since the latent code $z$ is unobserved, training this model would require optimizing the
evidence $p(x) = \int p(x,z)$.
However, due to the MLP parameterized $\mu_x$, the integral is intractable.
Variational inference performs approximate inference by introducing variational distribution
$q_\phi(z \mid x)$ and maximizing the following lower bound on $\log p(x)$:
\begin{equation}
    \log p(x) - \KL{q(z \mid x) || p(z \mid x)}
    = \Es{q_\phi(z \mid x)}{\log \frac{p_\theta(x,z)}{q_\phi(z\mid x)}} = \mcL(\theta, \phi).
\end{equation}

(Write out objective in full.)

While SVI introduces local parameters for each instance of $z$,
and AVI uses a single $q(z \mid x)$ for all instances,
we will follow the approach of SAVI.
We will perform inference as follows:
For each instance $x$, produce local variational parameter
$z^{(0)} = g(x; \phi)$.
Obtain $z^*$ by solving $\mcL(\theta, z^{(0)}) = 2$, with (local) optima $\ell^*$.
Take gradients through the whole procedure,
i.e. compute $\frac{\partial \ell^*}{\partial \phi}
= \frac{\partial\ell^*}{\partial z^*}\frac{\partial z^*}{\partial z^{(0)}}
\frac{\partial z^{(0)}}{\partial \lambda}$.
The main difficuty lies in computing $\frac{\partial z^*}{\partial z^{(0)}}$.
(Highlight challenge)

In order to avoid the memory costs of storing all intermediate computation performed
in a solver, we will instead apply the IFT.
In order to apply the IFT, we must satisfy the three conditions.
First, we must have a solution point to a system of equations, $F(x_0, z_0) = 0$.
In this setting, we will use the KKT conditions of the optimization problem to define $F$.


\bibliography{bib}

\appendix

\section{Example Appendix}
\label{sec:appendix}

Neural ODEs use reversibility.

\end{document}
